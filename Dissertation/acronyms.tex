\chapter*{Список сокращений и условных обозначений} % Заголовок
\addcontentsline{toc}{chapter}{Список сокращений и условных обозначений}  % Добавляем его в оглавление
% при наличии уравнений в левой колонке значение параметра leftmargin приходится подбирать вручную
\begin{description}[align=right,leftmargin=3.5cm]
\item[АИС] автоматизированная измерительная система
\item[АТС] автономное (беспилотное) транспортное средство
\item[ИК] инфракрасный (диапазон)
\item[ЛВС] локальная вычислительная сеть
\item[ПЛК] программируемый логический контроллер
\item[ПО] программное обеспечение
\item[СПФ] сухой пленочный фоторезист
\item[СТЗ] система технического зрения
\item[ЧПУ] числовое программное управление
\item[УП] управляющая программа
\item[BSD] Berkeley Software Distribution
\item[GPL] General Public License
\item[GPU] Graphics Processing Unit, блок вычисления графики
\item[JSON] JavaScript Object Notation
\item[OEE] Overall Equipment Efficiency, показатель общей эффективности оборудования.
\item[QR] Quick Response (code)
\item[RDP] Remote Desktop Protocol, протокол для удалённого доступа к компьютеру посредством ЛВС от Microsoft
\item[SDK] Software Developer's Kit, набор для разработки ПО.
\item[SIFT] Scale-Invariant Feature Transform
\item[VNC] Virtual Network Computing, протокол для удалённого доступа к компьютеру посредством ЛВС от Oracle
\end{description}
