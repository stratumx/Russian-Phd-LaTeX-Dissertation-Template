\chapter{Теоретические основы построения систем машинного зрения} \label{chapt3}

\section{Теория распознавания образов} \label{sect3_1}

В целом, можно выделить следующие методы распознавания образов:

\begin{enumerate}
	\item Метод перебора. В этом случае производится сравнение с базой данных, где для каждого вида объектов представлены всевозможные модификации отображения. Например, для оптического распознавания образов можно применить метод перебора вида объекта под различными углами, масштабами, смещениями, деформациями и т. д. Для букв нужно перебирать шрифт, свойства шрифта и т. д. В случае распознавания звуковых образов, соответственно, происходит сравнение с некоторыми известными шаблонами (например, слово, произнесенное несколькими людьми).
	\item Второй подход - производится более глубокий анализ характеристик образа. В случае оптического распознавания это может быть определение различных геометрических характеристик. Звуковой образец в этом случае подвергается частотному, амплитудному анализу и т. д.
	\item Следующий метод - использование искусственных нейронных сетей (ИНС). Этот метод требует либо большого количества примеров задачи распознавания при обучении, либо специальной структуры нейронной сети, учитывающей специфику данной задачи. Тем не менее, его отличает более высокая эффективность и производительность. [10].
	\item Экспертный метод, основанный на непрерывном обучении экспертной системы в процессе эксплуатации.
\end{enumerate}

\section{Основные признаки целевых объектов} \label{sect3_2}

\section{Анализ и синтез алгоритмов обработки изображений} \label{sect3_3}

\subsection{Общие положения задач}

\subsection{Базовые задачи, выполняемые трехкоординатной платформой}

\subsection{Расширенные задачи, требующие дополнительных модулей для СТЗ}

\subsection{Унификация алгоритмов}

\section{Оптические системы, применяемые в системах машинного зрения} \label{sect3_4}

\section{Выводы по третьей главе} \label{sect3_5}