\chapter{Проблематика исследования и постановка задачи} \label{chapt1}

\section{Общие положения о системах технического зрения} \label{sect1_1}

Техническое зрение является обширной отраслью науки и техники, имеющей широкое применение в практически всех областях деятельности человека. Год от года развитие этой отрасли лишь растёт, появляются новые прикладные задачи, выполняется адаптация существующих систем к новым условиям. Своё применение техническое зрение нашло в медицине, в космонавтике, в бытовых задачах вроде сканирования товара на кассе магазина. Беспилотные транспортные средства, системы безопасности на дорогах и в опасных зонах, системы наблюдения пассажиропотока в метро также не могут обойтись без применения технического зрения. И, разумеется, огромный пласт задач решается техническим зрением в промышленном производстве, от управления сборочными роботизированными комплексами до контролирующих операций. Уровень вовлечённости информационных технологий в производства очень высок, количество учитываемых и обрабатываемых данных растёт, что сказывается на качестве продукции и скорости её производства.

Несмотря на то, что техническое зрение появилось ещё в 60-е, полноценно применяться оно начало лишь 30 лет спустя. Началом данного раздела науки можно считать 1958 год, когда Фрэнк Розенблатт, профессор психологии из Корнеллской лаборатории аэронавтики, смог описать модель перцептрона, упрощенной модели человеческого восприятия. Перцептрон представлял собой сеть из передающих сигналов трех видов. Первые, сенсорные или S-элементы, отвечали за восприятие раздражителя, переходя из состояния покоя в состояние возбуждения. Вторые, ассоциативные или А-элементы, активизировались при преодолении некоего порога возбужденных S-элементов и ассоциировали этот набор с каким-то значением. Наконец, третий, реагирующий или R-элемент, представлял собой сумматор значений со всех А-элементов со своими весовыми коэффициентами. Перцептрон мог обучаться на изменении весовых коэффициентов значений ассоциативных сигналов. Два года спустя, в 1960 году, перцептрон был реализован аппаратно и получил название Mark I Perceptron. Чувствительная матрица компьютера состояла из 400 элементов (20 на 20 элементов) и могла выполнять несложные задачи распознавания, например, букв и цифр.

В 60-е гг. применение методов технического зрения было ограничено аэрокосмической отраслью. Между США и СССР шла космическая гонка, требующая высокой точности расчётов и совершенствования компьютерных технологий. Техническое зрение применялось при обработке телевизионных снимков со спутников, в задачах навигации и поиска посадочных площадок на местности. В это время, а также в последующие 70-е гг. шло активное нарабатывание методов обработки изображений и поиска интересующих объектов на них. Скачок развития технического зрения пришёлся на 80-е и 90-е гг., когда, во-первых, стала доступна транзисторная электроника, позволяющая изготавливать компактные микропроцессоры, а, во-вторых, появились первые относительно быстрые интерфейсы передачи данных. В 1996 году был представлен интерфейс Universal Serial Bus (USB), который, обладая пропускной способностью до 12 Мбит в секунду, позволял проводить обработку видеопотока в реальном времени. Параллельно разрабатывались и методы сжатия потокового цифрового видео, которые применялись в телевещании с 1984 года. 

В настоящее время новейшим стандартом сжатия видеопотока является H.266 (Versatile Video Coding), принятый летом 2020 года и обеспечивающий сжатие, по оценкам, вплоть до 16~\% от стандарта MPEG-4 двадцатилетней давности. Однако, наиболее рапространенным на сегодняшний день пока что остается H.264 (Advanced Video Coding) 2003 года, поскольку для него существует большое число кодеков, программного обеспечения и библиотек. В случае с более новыми стандартами, уровень развития программного обеспечения пока сравнительно низкий, кроме того, многие разработчики не решаются внедрять новые стандарты ввиду слишком громоздкой архитектуры приложений и сопутствующих этому затрат.

Техническое зрение объединяет в себе математические методы, информационные технологии и аппаратную составляющую.

\section{Текущее положение систем технического зрения в промышленности} \label{sect1_2}

\section{Уровень разработанности темы в исследованиях} \label{sect1_3}

\section{Модульный подход в проектировании оборудования с ЧПУ} \label{sect1_4}

\section{Описание разрабатываемой модульной трёхкоординатной платформы} \label{sect1_5}

\section{Выводы по первой главе} \label{sect1_6}
