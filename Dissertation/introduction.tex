\chapter*{Введение}                         % Заголовок
\addcontentsline{toc}{chapter}{Введение}    % Добавляем его в оглавление

\newcommand{\actuality}{}
\newcommand{\progress}{}
\newcommand{\aim}{{\textbf\aimTXT}}
\newcommand{\tasks}{\textbf{\tasksTXT}}
\newcommand{\novelty}{\textbf{\noveltyTXT}}
\newcommand{\influence}{\textbf{\influenceTXT}}
\newcommand{\methods}{\textbf{\methodsTXT}}
\newcommand{\defpositions}{\textbf{\defpositionsTXT}}
\newcommand{\reliability}{\textbf{\reliabilityTXT}}
\newcommand{\probation}{\textbf{\probationTXT}}
\newcommand{\contribution}{\textbf{\contributionTXT}}
\newcommand{\publications}{\textbf{\publicationsTXT}}


{\actuality} За долгие годы развития промышленного производства было создано множество способов автоматизации производственного процесса. Техническое или машинное зрение уже долгое время успешно применяется в широком спектре отраслей, среди которых существенную долю практического применения занимают задачи промышленного производства. С развитием информационных технологий существует необходимость делегирования от человека к техническим средствам и алгоритмам выполнение разного рода операций, требующих непрерывного контроля и высокой точности. Системы технического зрения или машинное зрение в общем виде --- это комплекс технических и программных средств, созданный с целью интерпретирования, классификации и определения объектов реального мира и использование этого знания для решения конкретной задачи. К этому же определению можно отнести совокупность методических средств и теоретических документов, описывающих данное направление с точки зрения функционирования и применяемости. Основным вектором применения машинного зрения можно назвать такие отрасли, как робототехника и производство, хотя схожие технологии можно встретить и в повседневной жизни, например, при сканировании штрих-кода в супермаркете или так называемые QR-коды \cite{Hung2020}. Суть работы такого оборудования в большинстве случаев неизменна: существует реальный объект, из которого путем визуального анализа необходимо получить некую информацию и соответствующим образом ее обработать. К примеру, в штрих-коде такой информацией будет страна и код изготовителя товара, в QR-коде будет содержаться ссылка на какую-либо страницу в Интернете или же простое текстовое сообщение.

В последнее время с развитием концепции Индустрии 4.0 \cite{Hermann2015, Moore2019, Schwab2017} отмечается рост использования вычислительных технологий в производственных процессах. Все большую популярность обретают небольшие производства, выпускающие единичные или малые партии продукции и способные быстро переориентировать производство под нужды заказчиков. В связи с этим появилась концепция модульных систем управления оборудованием с ЧПУ, более подходящая для нужд производителей \cite{Utin, Morales2010}. Модульный подход обеспечивает три важные характеристики оборудования: масштабируемость, гибкость и автономность. В соответствии с вышесказанным, машинное зрение также может существовать в систему управления в виде отдельного модуля, расширяемого или упрощаемого в соответствии с типом выполняемой задачи.

Как показывает практика, применение систем машинного зрения в промышленности позволяет существенно повысить эффективность производственных линий, повысить качество выпускаемой продукции, удешевить конечный продукт за счёт высвобождения людских ресурсов, а также выполнять мониторинг процесса, сигнализируя о внештатных или аварийных ситуациях. Кроме того, методами визуального анализа можно проводить первичный контроль готовой продукции, к примеру, определять чистоту поверхности по блику света на ней \cite{Allan2019}. Но, разумеется, в этом случае всё упирается в качество используемых технических средств и условия работы системы.

Программная часть системы машинного зрения является совокупностью методов обработки изображений и методов распознавания образов, поскольку в ходе распознавательного процесса необходимо не только «увидеть» объект, но и соответствующим образом классифицировать его. Задача классификации строится из возможности системы выделять признаки анализируемого объекта и отнесения его к какому-либо классу или подклассу. Чаще всего системы распознавания образов проектируются с вводом некоторого количества образов, классификация которых уже известна, таким образом системе проще распознавать объекты, сравнивая их с так называемым эталоном \cite{Kudr-92, Mest2004}. В общем виде это выглядит так: система принимает на вход объект, проводит анализ и выделяет характеристики, сравнивая их с тем, что было заложено в её базу знаний на этапах разработки, после чего, на основе полученных характеристик, запускается механизм присваивания объекта одному или нескольким классам, что является результатом, впоследствии используемым для решения других задач. Каждый этап работы системы должен содержать механизм обратной связи, чтобы база знаний могла дополняться и улучшаться. 

\textbf{Степень разработанности темы.} Разработками в области технического зрения занимаются многие специалисты из разных научных и исследовательских центров в разных странах. Работы в области промышленного производства составляют лишь малую часть разработок, но всё же она весьма обширна, что подчёркивает актуальность выбранной тематики. Это связано, в первую очередь, с появлением на рынке дешёвых и качественных комплектующих, к примеру, высокопроизводительные одноплатные компьютеры Odroid, Raspberry и Arduino, позволяющие реализовывать обрабатывающие алгоритмы. Также причиной служит переориентация промышленности на гибкие и масштабируемые системы управления оборудованием.

К примеру, исследовательская команда из Орхусского университета разрабатывают систему ЧПУ для лазерного резака на основе комбинирования маркеров и дополненной реальности \cite{6935435}. Система основана на подходе <<What You See Is What You Get\footnote{Что видите, то и получаете (пер. с англ.) --- концепция построения программного обеспечения, в которой заранее визуализируется то, что получит пользователь на выходе, а редактирование происходит в реальном времени.}>>, где проектор используется для отображения текущих контуров, а маркеры используются для установки его положения в рабочей области. Наряду с этим, специалисты университета Кейо расширяют функциональность фидуциальных  маркеров для лазерного резака \cite{071ddc0cfd994362b7c96f0ecbc300f3}. Чтобы установить параметры резки, они размещают набор маркеров опорных точек рядом с заготовкой, в том числе метки, связанные с материалом, порядком операций и командами. В работе описан способ обнаружения ошибок контура на основе машинного зрения. Разработан специальный измерительный прибор с нанесёнными на него маркерами, который позволяет исследователям измерять погрешность контура без сетчатого датчика. Задача анализа емкостей с жидкостями для поиска примесей с применением машинного обучения описывается в работе \textit{Ли Синью} и коллектива из тайваньского университета \cite{Li2018}.

Что касается применения машинного рения в промышленности, то на эту тему также публикуется большое число статей и книг. К примеру, книга
<<Manufacturing, Engineering \& Technology>> \textit{С. Калпакяна} посвящена вопросам «умного» производства с применением вычислительных технологий, в том числе, СТЗ \cite{Kalpakjian}. Группа исследователей под руководством \textit{Р. Усаментьяги} в своей статье приводят описание методик калибрования камеры для минимизации ошибок измерений, также описывается процесс производства калибровочных шаблонов \cite{Usamentiaga2017}. Немалое внимание также уделяется проблеме контроля изделий или операций в промышленности. Статья \textit{Оберга и Силкстрёма} описывает применение систем машинного зрения для контроля качества сварных швов с использование инфракрасных камер \cite{Oberg2017}. Ещё одна группа исследователей из Китая также представляют методику анализа поршня для двигателя внутреннего сгорания, с определением мелких дефектов. В частности, здесь качество поверхности анализируется по блику света на ней \cite{Xu2017}.

Разумеется, во введении сложно отразить то множество работ, которое посвящено тематике применения СТЗ в промышленности. Данному вопросу будет посвящён раздел \cite{sect1_3} настоящей диссертации, где будут отражены работы с 80-х годов прошлого века до наших дней. 

\textbf{Объектом исследования} является многоцелевое модульное технологическое оборудование с числовым программным управлением.

\textbf{Субъектом исследования} является система машинного зрения как совокупность технических и программных средств, направленных на обеспечение функциональности модульного оборудования в рамках решения задач, связанных с производственными процессами.

\textbf{Целью} работы является разработка модуля машинного зрения и его внедрение в систему управления модульной трёхкоординатной платформы.

Для~достижения поставленной цели необходимо было решить следующие {\tasks}:
\begin{enumerate}[beginpenalty=10000] % https://tex.stackexchange.com/a/476052/104425
 \item Анализ существующих работ по тебе промышленного применения средств машинного зрения, а также существующих патентов.
 \item Исследование предметной области: функции системы машинного зрения, условия рабочей среды, требуемых точностных и скоростных характеристик.
 \item Исследование влияния применения систем машинного зрения на производительность оборудования.
 \item Разработка методов повышения фактической производительности оборудования.
 \item Разработка состава технических средств, применяемых в комплексе.
 \item Разработка функциональной и структурной схемы компонентов системы.
 \item Разработка и реализация распознающих и измерительных алгоритмов.
 \item Тестирование системы машинного зрения и отдельных её компонентов во время процесса производства изделия.
 \item Анализ полученных результатов в плоскости совершенствования выполнения технологического процесса производства изделия по различным критериям.
 \item Внедрение комплекса в состав системы управления модульного оборудования с ЧПУ.
\end{enumerate}

\textbf{Научная и практическая значимость работы} отражена в применении данной системы в составе системы управления оборудования для производства печатных плат. Данная специфика характеризуется тем, что при производстве печатных плат применяется множество разных операций: обработка резанием, химическое воздействие на обрабатываемую поверхность, УФ-обработка и другие — однако, не существует универсального и доступного решения, чтобы объединить эти операции в одном устройстве. Для производства плат единичного или мелкосерийного характера (к примеру, производство прототипов) необходимо закупать несколько видов специализированного оборудования или размещать заказ на фабриках Китая, что сопряжено с большими временными затратами.

\textbf{Методы исследования.} Для решения обозначенных научных и инженерных задач использовались следующие научные положения: методики обработки изображений, методы распознавания образов, оптика и оптические системы, процедурное и объектно-ориентированное программирование, технологии построения локаьных вычислительных сетей.

{\defpositions}
\begin{enumerate}[beginpenalty=10000] % https://tex.stackexchange.com/a/476052/104425
 \item Методики выполнения основных функций машинного зрения с достигаемыми точностными характеристиками.
 \item Структурная схема компонентов компонента машинного зрения, отражающая взаимодействие компонентов друг с другом, а также с внешней средой модульного оборудования..
 \item Анализ результатов, достигаемых разрабатываемыми алгоритмами.
\end{enumerate} % Характеристика работы по структуре во введении и в автореферате не отличается (ГОСТ Р 7.0.11, пункты 5.3.1 и 9.2.1), потому её загружаем из одного и того же внешнего файла, предварительно задав форму выделения некоторым параметрам

\textbf{Объем и структура работы.} Диссертация состоит из~введения,
\formbytotal{totalchapter}{глав}{ы}{}{},
заключения и
\formbytotal{totalappendix}{приложен}{ия}{ий}{}.
%% на случай ошибок оставляю исходный кусок на месте, закомментированным
%Полный объём диссертации составляет  \ref*{TotPages}~страницу
%с~\totalfigures{}~рисунками и~\totaltables{}~таблицами. Список литературы
%содержит \total{citenum}~наименований.
%
Полный объём диссертации составляет
\formbytotal{TotPages}{страниц}{у}{ы}{}, включая
\formbytotal{totalcount@figure}{рисун}{ок}{ка}{ков} и
\formbytotal{totalcount@table}{таблиц}{у}{ы}{}.
Список литературы содержит
\formbytotal{citenum}{наименован}{ие}{ия}{ий}.

\textbf{Первая глава} содержит общее положение систем технического зрения в деятельности человека, в том числе, и в промышленности, степень разработанности темы и общее описание трёхкоординатной платформы с ЧПУ, которая будет выступать в роли полигона для отработки методик, определяемых в данной диссертации.

Системы технического зрения (СТЗ) --- комплекс аппаратных и программных средств, разработанных для получения некоторых данных из окружающего мира посредством анализа визуальной информации. Техническое зрение зародилось ещё в середине XX века с описанием перцептрона, реализующего модель восприятия информации мозгом. За последующие десятилетия отрасль развивалась, но основной скачок пришёлся на 90-е годы, когда появились быстрые интерфейсы передачи данных (USB), а аппаратные компоненты стали относительно дешёвыми и распространёнными. К XXI веку отрасль стала достаточно распространённой во многих сферах промышленности, в том числе благодаря развитию программных пакетов для проектирования СТЗ и, следовательно, снижению её себестоимости. На данный момент отрасль является востребованной, с оборотом 1.13 миллиарда долларов за 2019 год, при этом, большая часть сфер применения СТЗ приходится на промышленное производство.

СТЗ нашли своё применение в разных задачах, среди которых можно выделить иденЖданов А.А.тификацию, распознавание объектов, отслеживание объекта, измерения, навигация роботов, измерения и сравнения и другие. Так, без штрихкода уже нельзя себе представить товарооборот. Стандартизация СТЗ регламентируется рядом стандартов, разработанных ассоциациями Северной Америки (AIA), Европы (EMVA) и Японии (JIIA). Основной определяющий стандарт --- EMVA 1288 --- регламентирует множество нюансом проектирования и тестирования СТЗ, от методик измерений до используемых протоколов передачи данных. Не стоят на месте и стандарты передачи потокового видео, которое уже добилось высокой степени сжатия без потерь качества, что благотворно сказывается на быстродействии и точности работы СТЗ.

Поскольку промышленность --- одна из отраслей, вынужденных находиться на острие прогресса, то, разумеется, тема применимости СТЗ в промышленности развивается уже достаточно давно. Степень автоматизированности труда является одним из признаков развитой экономики государства. В промышленности СТЗ широко применяются, в частности, в задачах дефектоскопии, в задачах сборки, в контролирующих операциях, в отслеживании заготовок на протяжении всего цикла производства, в вопросах безопасности и многом другом. Чаще всего СТЗ действует обособленно от производственной линии, выступая надстройкой над ней, и лишь в очень редких случаях СТЗ встраивается в используемое оборудование и выполняет задачи, получая данные по ЛВС. Существуют решения, повышающие универсальность СТЗ, в частности, полноценные комплекты СТЗ с корпусом и выносными камерами (например, от Delta Electronics, Axiomtek или AAEON), у которых есть возможность составления своей программы обработки по нужды производства, но такие системы имеют свои ограничения и являются проприетарными.

Очевидно, что уровень исследовательской работы в данной сфере является высоким. Начиная с 80-х годов прошлого века в СССР группой учёных, куда входили Титов В. С., Жаботинский Ю.Д., Жданов А.А. и другие, была определена концепция производственной СТЗ, состоящей из ряда компонентов, обобщённый алгоритм её работы и классификация основных типов СТЗ. С развитием аппаратных средств и доступности библиотек для разработки СТЗ появилось множество работ о применении СТЗ в каких-то промышленных задачах, например, работы Ц. Хуанга, Ж. Хаты, А. Домеля и других. К сожалению, подавляющее большинство работ имеет прикладной характер и нацелены на решение какой-либо узкоспециализированной задачи, не ставя своей целью создания концепции универсальной СТЗ для использования в широком круге производственных задач.

В качестве основы для разработки методик работы СТЗ используется трёхкоординатная платформа с ЧПУ, которая относится к классу модульного оборудования. Она предназначена для обработки широкой номенклатуры изделий через возможность быстрой смены инструмента и предназначена для небольших производств, которые занимаются производством малых партий изделий. Она представляет собой шасси, оснащённое шаговыми двигателями и имеющее рабочее пространство 500 на 500 мм. Каретка, проводящаяся в движение шаговыми двигателями, является основой для закрепления модулей, крепёж осуществляется электромагнитами. Для взаимосвязи модулей используется ЛВС платформы, которая содержит регистры с данными о модулях.

\textbf{Вторая глава} содержит понятие о производительности оборудования, методах его расчёта и о влияющих на него факторах, а также описывает пути его повышения с внедрением систем технического зрения на отдельные участки производственных линий.

очевидно, что зачастую на производствах станки с ЧПУ используются неэффективно. Ввиду повышения требований к качеству имеется спрос на высокоточное оборудование, однако, и время на обработку такого изделия соответственно повысится. И, если крупные производства могут себе позволить выбирать, вкладываться ли им в дорогостоящее широкофункциональное оборудование или же строить производственные линии на основе узкоспециализированных станков. Малые производства такой возможности лишены, им проще приобрести более универсальное оборудование, которое будет выполнять все виды обрабатывающих операций.

В условиях постоянно меняющегося производства эффективность и производительность оборудования оценить довольно сложно и обычно малые предприятия не ведут пообных расчётов, поскольку подготовка производства у них занимает больше времени, чем само производство. Под производительность станка обычно понимается <<штучная производительность>> --- количество произведённых изделий за единицу времени. Показатель производительности оборудования относится больше к экономическому типу параметров и на крупных предприятиях от его правильного расчёта зависит окупаемость производства. Производительность напрямую относится к показателю эффективности оборудования, которая в литературе носит аббревиатуру OEE (Overall Equipment Effectiveness)\footnote{пер. с англ. --- ``Общая эффективность оборудования''}. Разумеется, производительность не является постоянной величиной и изменяется со временем. Этому способствуют как систематические факторы, такие как износ оборудования, инфляция, плановые ремонты, уровень организации рабочего места, так и стохастические: человеческий фактор, логистика, колебания рынка.

Для получения оценки производительности оборудования следует рассмотреть его работу в динамике, например, в течение 2 месяцев. В этом случае будут понятны степени влияния факторов изменения производительности оборудования. Но для получения гипотетической оценки можно ограничиться анализом типового технологического процесса. В качестве такового был выбрал процесс производства партии переходных печатных плат типа SOP14/SSOP14 с разводкой под корпус DIP-14.

Техпроцесс предполагает создание партии плат в 729 штук, каждая из которых имеет размеры 18 мм на 18 мм. Каждая плата имеет 14 проводящих отверстий диаметром 0,8 мм и два крепёжных диаметром 2 мм, а также разводку на обеих сторонах платы. Рассматриваемый технологический процесс изготовления партии печатных плат насчитывает 100 операций, из них 37 установочных, 33 вспомогательных, 18 обрабатывающих и 12 контролирующих. Итоговое нормативное время изготовления партии из 729 печатных плат составляет 105450 секунд или приблизительно 29,2 часа. Таким образом, можно утверждать, что при текущем технологическом процессе установка способна выдавать 24,9 печатных плат в час.

Если подсчитать время работы установки, то есть, суммарное время обрабатывающих операций, то оно составит 81620 секунд или 22,6 часа. По отношению к общему времени это составляет 77,6 \%. Нетрудно заметить, что основная доля неучтённого времени приходится на промежуточные операции контроля изделий после каждого этапа обработки. Это расчёт так называемого идеального процесса производства, без учёта влияния факторов, описанных выше.

Очевидно, что основная доля простоя оборудования приходится на операции контроля, которые необходимо выполнять после каждого этапа производства. Тем не менее, можно рассмотреть такие способы снижения времени производства, как объединение операций, смещение операций внутри техпроцесса, а также одновременное выполнение двух операций за счёт установки двух инструментальных модулей на каретку, но все они являются ситуативными и не будут работать на большинстве техпроцессов. Поэтому было решено рассмотреть методики снижения времени производства за счёт внедрения системы технического зрения.

\textbf{Третья глава} описывает теоретические основы построения систем машинного зрения, в том числе используемые в работе методы распознавания образов и обработки изображений.

Задача построения СТЗ даже под конкретную узкоспециализированную функцию всегда является комплексной и требует своего подхода. При разработке важно не только обеспечить нужный результат работы системы, но и обеспечить ее масштабируемость и технологичность. Под масштабируемостью здесь понимается готовность системы к расширению функционала в изменившихся условиях работы, а под технологичностью --- простую и понятную взаимосвязь структурных компонентов системы, а также минимальные затраты на её интеграцию и наладку. Проектирование любой СТЗ начинается с постановки задачи, определения условий работы, а также составления библиотеки распознаваемых объектов, так называемого <<датасета>>. Датасет важен в первую очередь для составления списка признаков искомого объекта и выявления общности этих признаков.

Существует ряд методов поиска объекта на изображениях:

\begin{enumerate}
	\item Метод сопоставления с шаблоном (перебора).
	\item Анализ параметров объекта.
	\item Нейросетевой подход.
\end{enumerate}

Последний подход является широко распространённым для создания детектирующих СТЗ в настоящее время, однако, в условиях постоянной смены изделий его применение не является целесообразным по причине того, что обучение нейросети займёт большое количество времени и будет проводиться постоянно по мере появления новых объектов для обнаружения.

\textbf{Четвертая глава} посвящена анализу полученных результатов работы, включая оценочные характеристики получаемых значений, а также дальнейшим перспективам разрабатываемого направления.
